\documentclass[uplatex, dvipdfmx, cjk, aspectratio=43]{beamer}
% \documentclass[uplatex, dvipdfmx, cjk, aspectratio=169, handout]{beamer}
\graphicspath{{./figure/}}
\usepackage{"./setting/slide"}
\usepackage{"./setting/macros"}

\begin{document}

\title{進捗報告}
\subtitle{ほげほげほえ}
\author{name}
\institute{hoge研究室}
\date{2019-11-13}

% ---------------------------

\begin{frame}
    \titlepage
\end{frame}

% ---------------------------

\begin{frame}{目次}
    \tableofcontents
\end{frame}

% ---------------------------

\begin{frame}{校了する場合}
    \begin{itemize}
        \item あ
        \item い
        \item う
    \end{itemize}
\end{frame}

% ---------------------------

\section{計算時間による遅延のみ考慮する場合}
\begin{frame}
    \begin{itemize}                %% 箇条書き
        \item 豊科\pause           %% \pauseが挿入された箇所でスライドを分割する
        \item 明科
    \end{itemize}

    % \begin{figure}
    %     \centering
    %     \includegraphics<1|handout:0>[width=8cm]{./figure/sys1.png}
    %     \includegraphics<2|handout:0>[width=8cm]{./figure/sys1-2.png}
    %     \includegraphics<3|handout:0>[width=8cm]{./figure/sys1-3.png}
    %     \includegraphics<4->[width=8cm]{./figure/sys2.png}
    % \end{figure}

    \begin{block}{これからの幾何学}
    \end{block}
    \begin{alertblock}{PowerPointの問題点}
    \end{alertblock}
    \begin{exampleblock}{評価関数}
    \end{exampleblock}


    \begin{theorem}[Fermat]
        定理型環境が使える.\\
        $a^{p-1} \equiv 1 \pmod{p}$
    \end{theorem}

\end{frame}

% ---------------------------

\begin{frame}[allowframebreaks]
    ~\cite{kashima2019}
    ~\cite{ito2018}
    \bibliographystyle{./setting/sie}
    \bibliography{refs}
\end{frame}

% ---------------------------

\end{document}
