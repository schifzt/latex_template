% \documentclass[10pt, a4paper, uplatex]{jsarticle}
% \documentclass[10pt, a4paper, draft, uplatex]{jsarticle}                         % disable images
% \documentclass[twocolumn, 8pt, a4paper, uplatex]{jsarticle}                      %2column

% ブロックコメントアウト
\iffalse
\fi


\begin{document}
\maketitle

\tableofcontents
% \columnseprule=0. 2mm

% multiple alignment
\begin{alignat}{3}
    & a && b && c \\
    & a && b && c 
\end{alignat}

% 連立方程式
\begin{align}
    \left\{
        \begin{array}{llll}
            i_2=\frac{v_2}{R_2} \\
            i_3=\frac{v_3}{R_3} \\
        \end{array}
    \right.
\end{align}


% 行列
\begin{pmatrix}
    a \\
    b
\end{pmatrix}

% プログラム
\begin{lstlisting}[language=R]
    library(rstan)
    fit <- stan(file = 'binomial.stan',
                data = data_dat,
                iter = 3000,
                warmup = 1000,
                chains = 4,
                seed = 1,
                thin = 2)
\end{lstlisting}




% figure x1
\begin{figure}
    \centering
    \includegraphics[width=15cm]{./kadai1/fig3.png}
    \caption{反転増幅回路}
    \label{fig3}
\end{figure}

% figure x1
\begin{figure}[h]
    \centering
    \begin{tikzpicture}
        \tikzincludegraphics[width=18cm]{}
        \xlabel{\small }
        \ylabel{\small }
    \end{tikzpicture}
    \caption{
    }
    \label{cum_r}
\end{figure}






% figure x3
\begin{figure}
    \centering
    \includegraphics[width=15cm]{./kadai1/fig1.png}
    \caption{反転増幅回路}
    \label{fig1}

    \includegraphics[width=15cm]{./kadai1/fig2.png}
    \caption{反転増幅回路}
    \label{fig2}

    \includegraphics[width=15cm]{./kadai1/fig3.png}
    \caption{反転増幅回路}
    \label{fig3}
\end{figure}


% figure x3 each caption
\begin{figure}[htb]
    \begin{minipage}[b]{0.45\linewidth}
        \centering
        \includegraphics[keepaspectratio, scale=0.5]{./kadai1/fig12}
        \subcaption{キャプション1}
        \label{l1}
    \end{minipage}
    \begin{minipage}[b]{0.45\linewidth}
        \centering
        \includegraphics[keepaspectratio, scale=0.5]{./kadai1/fig13}
        \subcaption{キャプション2}
        \label{l2}
    \end{minipage}
    \caption{
        ラピュタ終盤におけるムスカの態度の変化.
        \subref{l1}調子づくムスカ.
        \subref{l2}絶望の縁である.
    }
    \label{musuka}
\end{figure}


% itembox
\begin{itembox}[l]{最小自乗解}
    hogehoge
\end{itembox}

\renewcommand{\bibname}{参考文献}
\bibliographystyle{sieicej}
\bibliography{refs}
\end{document}
