% \documentclass[10pt, a4paper, uplatex]{jsarticle}
% \documentclass[10pt, a4paper, draft, uplatex]{jsarticle}                         % disable images
% \documentclass[twocolumn, 8pt, a4paper, uplatex]{jsarticle}                      %2column

% ブロックコメントアウト
\iffalse
\fi


\begin{document}
\maketitle

% \setcounter{section}{5}
\section{}

\subsection{}
\subsection*{\rule{0.7em}{0.7em} 黒四角を見出しのアクセントにする}


\tableofcontents
% \columnseprule=0. 2mm

% カッコの途中で改行
%   \leftは改行を認めないため,\right .(ピリオド)で改行前に疑似ペアをおいて閉じる.\rightも同様
%   https://stackoverflow.com/questions/36668334/right-doesnt-work-on-latex-multiline-equation
\begin{align}
    & \biggl\{ \biggr .
    a \\
    & b \\
    \biggl . \biggr\}
\end{align}
% subscriptの中で改行したいときは,\substack{}でくくる
\begin{align}
    \min_{\substack {a \\ b }}
\end{align}

% multiple alignment
% && を使うと全部左揃えになる.https://tex.stackexchange.com/questions/159723/what-does-a-double-ampersand-mean-in-latex
\begin{align}
    & a && b && c \\
    & a && b && c 
\end{align}
% \begin{alignat}{3}
%     & a && b && c \\
%     & a && b && c 
% \end{alignat}

% 連立方程式
\begin{cases}
    
\end{cases}

% 行列
\begin{pmatrix}
    a \\
    b
\end{pmatrix}

% プログラム
\begin{lstlisting}[language=R]
    library(rstan)
    fit <- stan(file = 'binomial.stan',
                data = data_dat,
                iter = 3000,
                warmup = 1000,
                chains = 4,
                seed = 1,
                thin = 2)
\end{lstlisting}




% figure x1
\begin{figure}
    \label{fig3}
    \centering
    \includegraphics[width=15cm]{./kadai1/fig3.png}
    \caption{反転増幅回路}
\end{figure}

% figure x1
\begin{figure}[h]
    \label{cum_r}
    \centering
    \begin{tikzpicture}
        \tikzincludegraphics[width=18cm]{}
        \xlabel{\small }
        \ylabel{\small }
    \end{tikzpicture}
    \caption{
    }
\end{figure}

% figure x2, horizontal
\begin{figure}[t]
    \centering
    \begin{tabular}{cc}
        \begin{minipage}[t]{0.45\hsize}
            \centering
            \begin{tikzpicture}
                \tikzincludegraphics[width=5cm]{lambda_max_vs_alpha.pdf}
                \xlabel{$d$}
                \ylabel{$\lambda_1^+,~ \lambda_1$}
            \end{tikzpicture}
        \end{minipage}
        
        \begin{minipage}[t]{0.45\hsize}
            \centering
            \begin{tikzpicture}
                \tikzincludegraphics[width=5cm]{m_vs_alpha.pdf}
                \xlabel{$d$}
                \ylabel{$\lvert m^+ \rvert,~ \abs{m}$}
            \end{tikzpicture}
        \end{minipage}
    \end{tabular}
    \caption{
        $\alpha$が$2,4,8$の値をとる場合の数値実験の結果.$\gamma$の値は$4.0$に固定した.
        赤実線は式(\ref{bbp-lambda}),(\ref{bbp-m})を表し,青実線は式(\ref{extr}),(\ref{saddle-point})から計算した解析解を表す.
        破線は各$\alpha,\gamma$における$d_c^+$と$d_c$をそれぞれ表す.
        }
        \label{flex_alpha}
\end{figure}




% figure x3
\begin{figure}
    \centering
    \includegraphics[width=15cm]{./kadai1/fig1.png}
    \caption{反転増幅回路}
    \label{fig1}

    \includegraphics[width=15cm]{./kadai1/fig2.png}
    \caption{反転増幅回路}
    \label{fig2}

    \includegraphics[width=15cm]{./kadai1/fig3.png}
    \caption{反転増幅回路}
    \label{fig3}
\end{figure}


% figure x3 each caption
\begin{figure}[htb]
    \begin{minipage}[b]{0.45\linewidth}
        \centering
        \includegraphics[keepaspectratio, scale=0.5]{./kadai1/fig12}
        \subcaption{キャプション1}
        \label{l1}
    \end{minipage}
    \begin{minipage}[b]{0.45\linewidth}
        \centering
        \includegraphics[keepaspectratio, scale=0.5]{./kadai1/fig13}
        \subcaption{キャプション2}
        \label{l2}
    \end{minipage}
    \caption{
        ラピュタ終盤におけるムスカの態度の変化.
        \subref{l1}調子づくムスカ.
        \subref{l2}絶望の縁である.
    }
    \label{musuka}
\end{figure}


% itembox
\begin{itembox}[l]{最小自乗解}
    hogehoge
\end{itembox}

\renewcommand{\bibname}{参考文献}
\bibliographystyle{sieicej}
\bibliography{refs}
\end{document}
